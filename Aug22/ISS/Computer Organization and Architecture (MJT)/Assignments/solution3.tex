\documentclass[12pt, letterpaper]{article}
\usepackage[margin=0.5in]{geometry}
\usepackage{amsmath}
\usepackage{amssymb}
\usepackage{fancyvrb}
\usepackage{minted}

\title{Assignment 3}
\author{Lokesh Mohanty (SR no: 21014)}
\date{August 2022}

\begin{document}

\maketitle

\section{Computer \& Compiler Details}
\label{sec:comp}

\subsection{Basic Information}
\label{sec:basic}
\begin{tabular}{@{$\bullet$ }ll}
  Architecture    &: x86\_64\\
  CPU op-mode(s)  &: 32-bit, 64-bit\\
  Address sizes   &: 48 bits physical, 48 bits virtual\\
  Byte Order      &: Little Endian\\
  CPU(s)          &: 16\\
\end{tabular}

\subsection{CPU Details}
\label{sec:cpu}
\begin{tabular}{@{$\bullet$ }ll}
  Vendor ID       &: AuthenticAMD\\
  Model name            &: AMD Ryzen 7 PRO 5875U with Radeon Graphics\\
  CPU family            &: 25\\
  Model                 &: 80\\
  Thread(s) per core    &: 2\\
  Core(s) per socket    &: 8\\
  Socket(s)             &: 1\\
  Stepping              &: 0\\
  Frequency boost       &: enabled\\
  CPU(s) scaling MHz    &: 44\%\\
  CPU max MHz           &: 4546.8750\\
  CPU min MHz           &: 1600.0000\\
\end{tabular}

\subsection{Cache}
\label{sec:cache}
\begin{tabular}{@{$\bullet$ }ll}
  L1d cache  &: 256 KiB (8 instances)\\
  L1i cache  &: 256 KiB (8 instances)\\
  L2 cache   &: 4 MiB (8 instances)\\
  L3 cache   &: 16 MiB (1 instance)\\
\end{tabular}

\subsection{Compiler Details}
\label{sec:compiler}

\begin{tabular}{@{$\bullet$ }ll}
  Compiler &: gcc (GCC)\\
  Version  &: 10.2.1 20201203\\
\end{tabular}

\section{Program Code}
\label{sec:code}

\subsection{Function to multiply 2 matrices}
\label{sec:code-matmul}

\begin{minted}{c}
void matmul(float C[][N], float A[][N], float B[][N]) {
  for (int i = 0; i < N; i++) {
    for (int k = 0; k < N; k++) {
      for (int j = 0; j < N; j++) {
        C[i][j] += A[i][k] * B[k][j];
      }
    }
  }
}
\end{minted}

\subsection{Function to verify correctness of matrix multiplication}
\label{sec:code-verify}

\begin{minted}{c}
bool verifyMatmul(float C[][N], float A[][N], float B[][N]) {
  for (int count = 0; count < 10; count++) {
    int i = rand() % N, j = rand() % N;
    float actualValue = 0;
    for (int k = 0; k < N; k++) actualValue += A[i][k] * B[k][j];
    if (C[i][j] != actualValue) return false;
  }
  return true;
}
\end{minted}

\subsection{Function to generate input matrices}
\label{sec:code-generate}

\begin{minted}{c}
void generateMatrix(float matrix[][N]) {
  for (int i = 0; i < N; i++) {
    for (int j = 0; j < N; j++) {
      matrix[i][j] = rand() % N;
    }
  }
}
\end{minted}

\subsection{Function to measure matrix multiplication time}
\label{sec:code-time}

\begin{minted}{c}
typedef void (*MatMul)(float [][N], float [][N], float [][N]);
void time(MatMul func, float C[][N], float A[][N], float B[][N]) {
  struct timeval start, end, timeTaken;

  gettimeofday(&start, NULL);
  func(C, A, B);
  gettimeofday(&end, NULL);

  bool isMatmulValid = verifyMatmul(C, A, B);
  printf("Is Valid Matmul: %s\n",
         isMatmulValid ? "true" : "false");

  timeTaken.tv_usec = end.tv_usec - start.tv_usec;
  timeTaken.tv_sec = end.tv_sec - start.tv_sec;
  if (timeTaken.tv_usec < 0) {
    timeTaken.tv_usec = 1000000 + timeTaken.tv_usec;
    timeTaken.tv_sec--;
  }
  printf("Time Taken: %ld.%lds\n",
         timeTaken.tv_sec, timeTaken.tv_usec);
}
\end{minted}

\subsection{Main Function}
\label{sec:code-main}

\begin{minted}{c}
int main() {
  // using static to prevent segmentation fault for large N
  static float A[N][N], B[N][N], C[N][N] = {0.0};

  generateMatrix(A);
  generateMatrix(B);
  time(matmul, C, A, B);
  return 0;
}
\end{minted}

\subsection{Imports and Macros}
\label{sec:code-imports}

\begin{minted}{c}
#include <stdio.h>
#include <stdlib.h>
#include <sys/time.h>
#define N 1024
\end{minted}

\subsection{Bash script for profiling}
\label{sec:code-profile}

\begin{minted}{c}
#!/bin/bash

OPTIONS=("-O2"
         "-O2 -floop-interchange"
         "-O2 -fpeel-loops -funroll-loops"
         "-O2 -floop-interchange -fpeel-loops -funroll-loops"
         "-O2 -ftree-vectorize -fopt-info-vec"
         "-O2 -floop-interchange -fpeel-loops -funroll-loops -ftree-vectorize -fopt-info-vec")

for option in "${OPTIONS[@]}"; do
    gcc $option Ass3.c
    echo $option
    # remove pagecache
    sync
    sudo sh -c 'echo 1 > /proc/sys/vm/drop_caches'
    # time for i in {1..10}; do ./a.out; done
    for i in {1..10}; do
        perf stat -e cycles,instructions,cache-misses,cache-references ./a.out;
    done
done
\end{minted}

\section{Results \& Observations}
\label{sec:res}

\subsection{Results}
\label{sec:res1}

With Basic optimizations\\
\textbf{Flags:} \verb~-O2~\\
\begin{tabular}{|l|r|}
  \hline
  Time& $0.388s$\\
  \hline
  Cycles& $1.693 \times 10^9$\\
  \hline
  Instructions& $7.531 \times 10^9$\\
  \hline
  Instructions/Cycle & $4.45$\\
  \hline
\end{tabular}\\

With loop interchange\\
\textbf{Flags:} \verb~-O2 -floop-interchange~\\
\begin{tabular}{|l|r|}
  \hline
  Time& $0.393s$\\
  \hline
  Cycles& $1.683 \times 10^9$\\
  \hline
  Instructions& $7.531 \times 10^9$\\
  \hline
  Instructions/Cycle & $4.47$\\
  \hline
\end{tabular}\\

With loop unrolling\\
\textbf{Flags:} \verb~-O2 -fpeel-loops -funroll-loops~\\
\begin{tabular}{|l|r|}
  \hline
  Time& $0.385s$\\
  \hline
  Cycles& $1.680 \times 10^9$\\
  \hline
  Instructions& $4.707 \times 10^9$\\
  \hline
  Instructions/Cycle & $2.90$\\
  \hline
\end{tabular}\\

With both loop interchange and unrolling\\
\textbf{Flags:} \verb~-O2 -floop-interchange -fpeel-loops -funroll-loops~\\
\begin{tabular}{|l|r|}
  \hline
  Time& $0.386s$\\
  \hline
  Cycles& $1.693 \times 10^9$\\
  \hline
  Instructions& $4.707 \times 10^9$\\
  \hline
  Instructions/Cycle & $2.79$\\
  \hline
\end{tabular}\\

With vectorization\\
\textbf{Flags:} \verb~-O2 -ftree-vectorize -fopt-info-vec~\\
\begin{tabular}{|l|r|}
  \hline
  Time& $0.097s$\\
  \hline
  Cycles& $0.373 \times 10^9$\\
  \hline
  Instructions& $2.172 \times 10^9$\\
  \hline
  Instructions/Cycle & $5.82$\\
  \hline
\end{tabular}\\

With loop interchange, loop unrolling and vectorization\\
\textbf{Flags:} \verb~-O2 -floop-interchange -fpeel-loops -funroll-loops -ftree-vectorize -fopt-info-vec~\\
\begin{tabular}{|l|r|}
  \hline
  Time& $0.091s$\\
  \hline
  Cycles& $0.395 \times 10^9$\\
  \hline
  Instructions& $1.465 \times 10^9$\\
  \hline
  Instructions/Cycle & $3.71$\\
  \hline
\end{tabular}\\

\subsection{Observations}
\label{sec:obs}

From the results we can see that we get the best optimization when we use a combination of loop-interchange, loop-unroll and vectorization optimization flags

\end{document}

%%% Local Variables:
%%% mode: latex
%%% TeX-master: t
%%% End:
