\documentclass[12pt, letterpaper]{article}
\usepackage[margin=0.5in]{geometry}
\usepackage{amsmath}
\usepackage{amssymb}
\usepackage{fancyvrb}
% \usepackage{minted}
\usepackage{graphicx}
% \graphicspath{ {./images/} }

\title{DS 221 Assignment: OpenMP Primality Test and MPI Matrix-Matrix Multiply}
\author{Lokesh Mohanty (lokeshm@iisc.ac.in)}

% insert a dot after the number and before the title
\renewcommand{\thesection}{\arabic{section}.}
\renewcommand{\thesubsection}{\arabic{section}.\arabic{subsection}.}

% give vertical padding to table cells
\renewcommand*\arraystretch{1.3}

\begin{document}

\maketitle

\section{Computer(turing) \& Compiler Details}
\label{sec:comp}

\subsection{Basic Information}
\label{sec:basic}
\begin{tabular}{@{$\bullet$ }ll}
  Architecture    &: x86\_64\\
  CPU op-mode(s)  &: 32-bit, 64-bit\\
  Byte Order      &: Little Endian\\
  CPU(s)          &: 12\\
\end{tabular}

\subsection{CPU Details}
\label{sec:cpu}
\begin{tabular}{@{$\bullet$ }ll}
  Vendor ID       &: GenuineIntel\\
  Model name            &: Intel(R) Xeon(R) CPU E5-2620 v3 @ 2.40GHz\\
  CPU family            &: 6\\
  Model                 &: 63\\
  Thread(s) per core    &: 1\\
  Core(s) per socket    &: 6\\
  Socket(s)             &: 2\\
  Stepping              &: 2\\
  NUMA node(s)          &: 2\\
  NUMA node0 CPU(s)     &: 0-5\\
  NUMA node1 CPU(s)     &: 6-11\\
  CPU max MHz           &: 3200.0000\\
  CPU min MHz           &: 1200.0000\\
\end{tabular}

\subsection{Cache}
\label{sec:cache}
\begin{tabular}{@{$\bullet$ }ll}
  L1d cache  &: 32 K\\
  L1i cache  &: 32 K\\
  L2 cache   &: 256 K\\
  L3 cache   &: 15360 K\\
\end{tabular}

\subsection{Compiler Details}
\label{sec:compiler}

\begin{tabular}{@{$\bullet$ }ll}
  Compiler(openmp) &: gcc (GCC) version 5.4.0 20160609\\
  Compiler(mpi)    &: MPICH version 3.2\\
\end{tabular}

\section{OpenMP Primality Test}

I implemented fermat primality test to find prime numbers from 1 to 500 in C.
Below is the implementation of the primality test.
Here for every number $n$, $a^{n - 1}\,mod\,n$ is computed for
all $a$ from $2$ to $n - 1$. If any of the values is not $1$ then the number $n$ is not a prime number.
Since $a^{n - 1}$ is greater than $2^{32}$ i.e., the limit of $int$, I am
calculating $a^{n - 1}\,mod\,n$ iteratively to prevent overflow.

\begin{verbatim}
// Returns '0' if false and '1' if true
char isPrime(int n) {
  if (n < 2) return '0';
  for (int i = 2; i < n-1; i++) {
    int r = i;

    for (int j = 2; j <= n - 1; j++)
      r = (r * i) % n;
    if (r != 1)
      return '0';
  }
  return '1';
}
\end{verbatim}

I am using OpenMP to parallelize the above function. To prevent race conditions, I am storing the result of the primality check in a array of size 500 and later printing the numbers who succeded in the primality check.

\begin{verbatim}
#pragma omp parallel for default(shared)
  for (int i = 1; i <= N; i++)
    prime[i] = isPrime(i);
\end{verbatim}

To use custom number of threads,
I am setting the environment variable \verb~OMP_NUM_THREADS~ to the number of threads.
I am using \verb~omp_get_wtime~ to measure the time taken.

\subsection{Results:}
\begin{tabular}{@{$\bullet$ }lll}
  1 Thread (sequential)  &: 0.074477 seconds &\\
  4 Threads              &: 0.061988 seconds &$\rightarrow$ 1.201  speedup \\
  8 Threads              &: 0.051333 seconds &$\rightarrow$ 1.451 speedup\\
  16 Threads             &: 0.046540 seconds &$\rightarrow$ 1.600 speedup\\
\end{tabular}

\section{MPI Matrix-Matrix Multiplication}

I implemented matrix matrix multiplication by dividing the the rows of matrix \textbf{A} among all the MPI processes i.e., $3200/64 = 50$ rows per block with all the columns of \textbf{A}. This block of \textbf{A} is passed to each process by maintaining an index/offset of the starting row number and the whole matrix \textbf{B}. Every process returns a block of $50 \times 3200$ rows combining which gives the whole matrix \textbf{C}. For passing data from master to slaves and from slaves to master I am using \verb~MPI_Send~ and \verb~MPI_Recv~ to receive data.
I am using \verb~MPI_Wtime~ to measure the time taken.

\begin{verbatim}
MPI_Send(&index, 1, MPI_INT, dest, mtype, MPI_COMM_WORLD);
MPI_Send(&rows, 1, MPI_INT, dest, mtype, MPI_COMM_WORLD);
MPI_Send(&a[index][0], rows*rowsMax, MPI_DOUBLE, dest, mtype, MPI_COMM_WORLD);
MPI_Send(&b, colsMax*colsMax, MPI_DOUBLE, dest,
       mtype, MPI_COMM_WORLD); index = index + rows;
\end{verbatim}

\begin{verbatim}
MPI_Recv(&index, 1, MPI_INT, source, mtype, MPI_COMM_WORLD, &status);
MPI_Recv(&rows, 1, MPI_INT, source, mtype, MPI_COMM_WORLD, &status);
MPI_Recv(&c[index][0], rows*colsMax, MPI_DOUBLE, source, mtype, 
MPI_COMM_WORLD, &status);
\end{verbatim}

\subsection{Results:}
\begin{tabular}{@{$\bullet$ }lll}
  1 Process(sequential)  &: 5 minutes 40 seconds &\\
  64 Processes           &: 18.456855 seconds &$\rightarrow$ 18.421 speedup\\
\end{tabular}

\end{document}


%%% Local Variables:
%%% mode: latex
%%% TeX-master: t
%%% End:
