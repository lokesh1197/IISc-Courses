% Created 2023-03-20 Mon 11:12
% Intended LaTeX compiler: pdflatex
\documentclass[11pt]{article}
\usepackage[utf8]{inputenc}
\usepackage[T1]{fontenc}
\usepackage{graphicx}
\usepackage{longtable}
\usepackage{wrapfig}
\usepackage{rotating}
\usepackage[normalem]{ulem}
\usepackage{amsmath}
\usepackage{amssymb}
\usepackage{capt-of}
\usepackage{hyperref}
\author{Lokesh Mohanty}
\date{\today}
\title{Game Theory Notes}
\hypersetup{
 pdfauthor={Lokesh Mohanty},
 pdftitle={Game Theory Notes},
 pdfkeywords={},
 pdfsubject={},
 pdfcreator={Emacs 28.2 (Org mode 9.7-pre)}, 
 pdflang={English}}
\begin{document}

\maketitle
\tableofcontents


\section{Properties required for auction (06/03/2023)}
\label{sec:orga2b49cf}
\begin{enumerate}
\item DSIC
\item Maximize social welfare (\(\sum x_i v_i\))
\item Computational tractability
\end{enumerate}

\section{Design Approach (06/03/2023)}
\label{sec:orgaf18c2d}
Reduce mechanism design to algorithm design

\textbf{Step  I}: Assume without justification, truthful bids design. (for properties 2 and 3)
\textbf{Step II}: Given \(x^{*}\), set \(p^{*}\) to achieve DSIC. (for property 1)

\section{Myerson's Lemma (06/03/2023)}
\label{sec:orge37cfb1}
For any single parameter environment
\begin{enumerate}
\item An allocation rule is implementable iff it is monotonous.
\item If x is monotonous, then there exists a unique payment rule(p) such that (x, p) is DSIC.
\item The payment rule (p) admits an explicit formula.
\end{enumerate}

\begin{align*}
p_i(b_i, b_{-i}) = \int_0^{b_i} z x_i' (z, b_{-i}) dz
\end{align*}

Monotonocity:
\(x\) is said to be monotonous \(\forall i\), \(\forall b_i\) if \(x_i(z, b_{-i})\) is non-decreasing in \(Z\)

\section{Single Parameter Environment (10/03/2023)}
\label{sec:orgf0f8b80}
n Bidder each with private \(v_i\) and quasi-linear utility

Allocatoin Rule \(x: \mathbb{R}_+^n \rightarrow X\)(feasible set)
Payment Rule \(p: \mathbb{R}_+^n \rightarrow \mathbb{R}_+^n\)

Error of Single parameter environment
\begin{itemize}
\item Single Item Auction
\begin{itemize}
\item DSIC
\item Welfare max \(\sum_i^n x_iv_i\), \(\left( \sum_i^n x_ib_i \right)\)
\item computational efficient
\end{itemize}
\item Ad Auction
\end{itemize}

Write \(x(z) = x_i(z, b_i)\) and \(p(z) = p_i(z, b_i)\)
\textbf{Proof}:
\begin{align*}
0 &\leq z < y \\
z x(z) - p(z) &\geq z x(y) - p(y) \\
y x(y) - p(y) &\geq y x(z) - p(z) \\
\implies z (x(y) - x(z)) &\leq p(y) p(z) \leq y (x(y) - x(z))
\end{align*}

for piece-wise constant \(x_i\),
\begin{align*}
p_i(b_i, b_{-i}) = \sum_{j=1}^l z_i\text{ jump }x_0(, b_{-i})
\end{align*}

general form:
\begin{align*}
p_i'(z, b_i) = z_i x_i'(z, b_i)
\end{align*}

\begin{center}
\includesvg[width=.9\linewidth]{./myer}
\end{center}

\textbf{Eg:} Single Item Auction
\(\hat{x}\) = second highest bidder -> not implementable as it is not monotonous

Myerson's lemma's insight: focus only on allocation rule and ignore payment rule.
\begin{itemize}
\item Ensure: monotonous, welfare maximization and computational efficient
\end{itemize}


\textbf{Eg:} Single Item single bidder (reserve price auctions) -> fixed price, the bidder gets it if he pays the price (reserve price)

\section{Revealation Principle (10/03/2023)}
\label{sec:orga8e7df5}
\textbf{Theorem 2}:
for every auction \(\mu\) in which every bidder has a dominant strategy (no matter the private valuation),
there exists an equivalent truth telling DSIC auction \(\mu'\)

\section{Bayesian Model (13/03/2023)}
\label{sec:org032f8ba}
\subsection{Revenue}
\label{sec:orgf3b2a48}
Revenue: \(\sum_{i=1}^n p_i(b)\)
Myerson's law doesn't work for Revenue maximization. Due to lack of a good model, we use bayesian model

\textbf{Example}: Single Item single bidder(reserve price auction) has trivial solution for welfare maximization but not for revenue maximization

\subsection{Model}
\label{sec:org4d5c0a3}
\begin{itemize}
\item Single parameter environment
\item Known distributions (\(F_i\))
\item Assume valuations (\(v_i \sim F_i\)) are mutually independent
\end{itemize}

Maximize Expectation: \(\mathbb{E}_{v \sim F} \left[ \sum_i p_i(v) \right]\), where \(F = F_1 \times F_2 \times ... \times F_n\)

Goal: Identify DSIC auction with max expected revenue

Expected Revenue for general r: \(\mathbb{E} \left[ p_1 (v) \right] = r \mathbb{P}_{F_i}\{ v_i \geq r \} = r(1 - F_i(r))\)

\textbf{Example}: Single Item 2 bidder
\(F_1 = F_2 = \text{Unif}[0, 1]\)
Expected Revenue of second price auction: \(\mathbb{E}_{v_1 \sim F_1, v_2 \sim F_2} [\sum_{i=1}^2 p_i(v)] = \mathbb{E}_{v \sim F} [min \{v_1, v_2\}] = 1/3\)

\textbf{Example}: Single Item 2 bidder with reserve price
Expected Revenue of second price auction:
\begin{align*}
\mathbb{E}_{v_1 \sim F_1, v_2 \sim F_2} \left[ \sum_{i=1}^2 p_i(v) \right] &= 0 . Pr\{v_1 < 1/2, v_2 < 1/2\} + 1/2 . Pr\{v_1 \geq 1/2, v_2 < 1/2\} \\
&+ 1/2 . Pr\{ v_1 < 1/2, v_2 \geq 1/2\} + \mathbb{E}[\min(v1, v2)]. Pr\{v_1 \geq 1/2, v_2 \geq 1/2\}
\end{align*}

\subsection{Revenue maximization using virtual welfare maximization (13/03/2023, 15/03/2023)}
\label{sec:orgca81fca}
Goal: Description of an expected revenue maximizing DSIC auction for any single parameter environment and any \(F_1, F_2, ..., F_n\)

Result(Myerson): Expected Revenue = Expected Virtual Welfare
Definition (Virtual welfare): For a bidder i, with value distribution \(F_i\) and valuation \(z_i\), their virtual value is defined as
\begin{align*}
\phi_i(z) = z - \frac{1 - F_i(z)}{f_i(z)}
\end{align*}
will assume \(F_1\) is continuous throughout, r support on \([0, v_{max}]\)
we can see from the definition that \(\phi_i(v_i) \leq v_i\)

\textbf{---------------------------------------------------- (15/03/2023)--------------------------------------------------------------}

\textbf{Lemma}: For every single parameter environment with valuation distributions \(F_1, F_2, ..., F_n\), every DSIC(x, p) mechanism.

Every \(i\), every \(v_{-i}\) of the other agents
\begin{align*}
\mathop{\mathbb{E}}_{v_i \sim F_i} [p_i(v)] = \mathop{\mathbb{E}}_{v_i \sim F_i} [x_i(v_i, v_{-i}) \phi_i(v_i)]
\end{align*}
\textbf{Proof}:
\begin{align*}
p_i(v_i, v_{-i}) &= \int_0^{v_i} z x_i'(z, v_{-i} dz \\
\mathop{\mathbb{E}}_{v_i \sim F_i} [p_i(v_i, v_{-i})] &= \int_0^{v_{max}} p_i (v_i, v_{-i}) f_i(v_i) dv \\
\implies \mathop{\mathbb{E}}_{v_i \sim F_i} [p_i(v)] &= \int_0^{v_{max}} \left[ \int_0^{v_i} z x_i'(z, v_{-i})dz \right] f_i(v_i) dv_i \\
&= \int_0^{v_{max}} \left[ \int_0^{v_{max}} f_i(v_i) dv_i \right] z x_i'(z, v_{-i})dz \\
&= \int_0^{v_{max}} \left[ 1 - F_i(z) \right] z x_i'(z, v_{-i})dz \\
&= \int_0^{v_{max}} \left[ 1 - F_i(z) \right] z x_i'(z, v_{-i})dz \\
&= \left[ 1 - F_i(z) \right]z \left( x_i'(z, v_{-i}) \right)\mathop{|}_0^{v_{max}} - \int_0^{v_{max}} x_i(z, v_{-i})(1 - F_i(z)  -zf_i(z))dz \\
&= 0 + \int_0^{v_{max}} x_i(z, v_{-i}) f_i(z) \left( z - \frac{1-F_i(z)}{f_i(z)} \right)dz \\
&= \int_0^{v_{max}} x_i(z, v_{-i}) \phi_i(z) f_i(z) dz \\
&= \mathop{\mathbb{E}}_{v_i \sim F_i} \left[ x_i(v_i, v_{-i}) \phi_i(v_i) \right]
\end{align*}

\textbf{Theorem}: For every single param environment with valuation distributions \(F_1, F_2, ..., F_n\) and every DSIC with \((x, p)\)
\begin{align*}
\mathop{\mathbb{E}}_{v \sim F} \left[ \sum_{i=1}^n p_i(v) \right] &= \mathop{\mathbb{E}}_{v \sim F} \left[ \sum_{i=1}^n x_i(v) \phi_i(v_i) \right] \\
\text{Where } F &= \mathop{\Pi}_{i=1}^nF_i
\end{align*}

\textbf{Proof}:
\begin{align*}
\mathop{\mathbb{E}}_{v \sim F} \left[ \sum_{i=1}^n p_i(v) \right] &= \mathop{\mathbb{E}}_{v \sim F} \left[ \sum_{i=1}^n \phi_i(v_i)x_i(v) \right] \\
\implies \mathop{\mathbb{E}}_{v \sim F} \left[ p_i(v) \right] &= \mathop{\mathbb{E}}_{v \sim F} \left[ \phi_i(v_i) x_i(v) \right] \text{  \{Linearity of expectation\}}
\end{align*}


\textbf{New Objective (equivalent)}: Identify DSIC mechanism \((x, p)\) that maximizes expected virtual welfare
\begin{align*}
\max \sum_{i=1}^n x_i \phi_i(v_i) \text{    s.t.   } (x_1, x_2, ..., x_n) \in \mathcal{X}
\end{align*}
This is called virtual-welfare maximizing allocation

\textbf{Sufficient} condition for monotonocity of \(x^{*}\)
\textbf{Definition} [Regular Distribution]: A distribution for \(F_i\) is said to be regular if the
corresponding virtual valuation function \(\phi_i(z) = z - \frac{1 - F_i(z)}{f_i(z)}\) is non-decreasing

Virtual Welfare maximiser \(\rightarrow\) Mechanism \(x^{*}\)

\uline{Assumptions}: All \(F_i\) are regular
\begin{enumerate}
\item Transform the (truthfully) reported valuations (\(v_i\)) into \(\phi_i(v_i)\)
\item Choose the feasible allocation \((x_1, x_2, ..., x_n) \in \mathcal{X}\) that maximizes \(\sum_{i=1}^nx_i(\phi_i(v_i))\)
\item Choose payment via Myerson's Lemma
\end{enumerate}

\textbf{Theorem}: [Virtual welfare maximizer \(x^{*}\) are revenue optimizers]
For every single parameter environemnts and regular distributions of the corresponding virtual welfare max \(x^{*}\) is a DSIC mechanism with max possible expected revenue.

\textbf{Eg:} Revenue optimum mechanism for single item auction
Consider \(v_1, v_2, ..., v_n \sim F_1\), \(F_1 \rightarrow\) strongly regular
Here \(x^{*}\) corresponds to
\begin{align*}
\max \sum_{i=1}^n x_i \phi(v_i) \text{  s.t.  } \sum_{i=1}^n x_i \leq 1
\end{align*}
If \(v_{(1)} > v_{(2)} > ... > v_{(n)} \implies \phi(v_{(1)}) > \phi(v_{(2)}) > ... > \phi(v_{(n)})\) 
Reserve price \((r) = \phi^{-1}(0)\)

Optimum: Second price auction with reserve price \(\phi^{-1}(0)\)

\textbf{---------------------------------------------------- (20/03/2023)--------------------------------------------------------------}

\textbf{Eg:} Single Item non-identical bidders (regular) \(v_i \sim F_i\) can have \(\phi_i(v_i) > \phi_2(v_2)\) with \(v_1 < v_2\)

This shows that revenue optimized auctions can become complex leading to wierd auctions which are not interpretable.

\subsection{Simple Near-Optimal Auctions}
\label{sec:orgec528af}
This give simple, robust and practical but near-optimal auction

\subsubsection{Prophet Inequality}
\label{sec:orgc14e930}
Known Distributions: \(G_1, G_2, ..., G_n\) (independent)

In each round \(i \in \{1, 2, ..., n\}\), observe \(\pi_i \sim G_i\)
After seeing \(\pi_i\), irrecovable decision: accept \(\pi_i\) and stop (or) discard \(\pi_i\) and continue
\textbf{Benchmark}: Fully clairvoyant prophet \(\rightarrow\) \(\mathbb{E}_{\pi \sim G} [\max_{i \leq i \leq n} \pi_i]\)

\textbf{Theorem:} For every sequence of n distributions (Samuel-Cahn) \(G_1, G_2, ..., G_n\), there exists a strategy that
guarantees expected reward at least \(\frac{1}{2} \mathbb{E}[\max \pi_{i}]\). However, there is such a threshold strategy which accepts
price \(\pi_i\)  iff \(\pi_i \geq\) some threshold \(t\)

\textbf{Proof:} Write \(Z^{+} := \max \{Z, 0\}\), \(q(t) =\) probability that all \(\pi_i < t\)

Define:
Event \(E_i\): only price \(\pi_i \geq t\) (\(\pi_j < t\, \forall j \neq i)\)


Derive an linear bound on expected reward of \(t\) threshold strategy:
\begin{align*}
\mathbb{E} \left[\text{strategy reward} \middle| E_i \right] &= \mathbb{E} \left[ \pi_i \middle| E_i \right] \\
&= t + \mathbb{E} \left[ \pi_i - t \middle| E_i \right]\\
\implies \mathop{\mathbb{E}}_{\pi \sim G} \left[ \text{strategy reward} \right]
&\geq (1 - q(t))t + \sum_{i=1}^n \mathop{\mathbb{E}}_{\pi \sim G} \left[ \pi_i - t \middle| E_i \right] Pr \{E_i\}  \\
&= (1 - q(t))t + \sum_{i=1}^n \mathop{\mathbb{E}}_{\pi \sim G} \left[ \pi_i - t \middle| \pi_i \geq t, \pi_j < t \, \forall j  \right] Pr \{\pi_i \geq t\} Pr \{\pi_j < t \, \forall j \neq i\}  \\
&= (1 - q(t))t + \sum_{i=1}^n \mathop{\mathbb{E}}_{\pi \sim G} \left[ \pi_i - t \middle| \pi_i \geq t \right] Pr \{\pi_i \geq t\} Pr \{\pi_j < t \, \forall j \neq i\}  \\
&= (1 - q(t))t + \sum_{i=1}^n \mathop{\mathbb{E}}_{\pi \sim G} \left[ \pi_i - t \middle| \pi_i \geq t \right] Pr \{\pi_i \geq t\} q(t)  \\
&= (1 - q(t))t +  q(t)\sum_{i=1}^n \mathop{\mathbb{E}} \left[ \pi_i - t \right]^{+}   \\
\implies \mathbb{E} \left[\text{strategy reward} \middle| E_i \right] &\geq (1 - q(t))t +  q(t)\sum_{i=1}^n \mathop{\mathbb{E}} \left[ \pi_i - t \right]^{+}
\end{align*}

Derive an upper bound on expected reward of prophet:
\begin{align*}
\mathop{\mathbb{E}} \left[ \max_i \pi_i \right] &= \mathop{\mathbb{E}}_{\pi} \left[ t + \max_{1 \leq i \leq n} (\pi_i - t) \right] \\
&= t + \mathop{\mathbb{E}}_{\pi} \left[ \max_i (\pi_i - t) \right] \\
&\leq t + \mathop{\mathbb{E}}_{\pi} \left[ \max_i (\pi_i - t)^+ \right] \leq t + \sum_{i=1}^n \mathop{\mathbb{E}}_{\pi} \left[ (\pi_i - t)^+ \right]
\end{align*}

Therefore for \(q(t) = \frac{1}{2}\) and from lower and upper bounds we can see that
\begin{align*}
\mathbb{E} \left[ \text{strategy reward} \right] \geq \frac{1}{2} \mathbb{E} \left[ \max_i \pi_i \right]
\end{align*}

\uline{Note:} work with adversarial tie breaking
Mechanism: virtual threshold
\begin{enumerate}
\item Choose L\textsuperscript{*} such that \(Pr \{ \max_i\}\)
\end{enumerate}
\section{Tasks}
\label{sec:orgba605b6}
\begin{itemize}
\item[{$\square$}] Assignment (Welfare maximization always leads to monotonous allocation rule)
\item[{$\square$}] Homework: \(X_1, X_2, ..., X_n \sim Unif[a, b]\), \(\mathbb{E} [\min_{1 \leq i \leq n} X_i] = \frac{b + na}{n + 1}\), \(Y = \min_{1 \leq i \leq n}X_{i}\)
\item[{$\square$}] Midterm 2: March 29
\end{itemize}

\section{Glossary}
\label{sec:orge09c61b}
\begin{itemize}
\item DSIC: Dominant Strategy Incentive Compatibility
\end{itemize}
\end{document}