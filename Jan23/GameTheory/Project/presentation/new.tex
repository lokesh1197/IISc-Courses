% Created 2023-04-10 Mon 10:20
% Intended LaTeX compiler: pdflatex
\documentclass[presentation]{beamer}
\usepackage[utf8]{inputenc}
\usepackage[T1]{fontenc}
\usepackage{graphicx}
\usepackage{longtable}
\usepackage{wrapfig}
\usepackage{rotating}
\usepackage[normalem]{ulem}
\usepackage{amsmath}
\usepackage{amssymb}
\usepackage{capt-of}
\usepackage{hyperref}
\usetheme{default}
\usecolortheme{}
\usefonttheme{}
\useinnertheme{}
\useoutertheme{}
\author{Lokesh Mohanty}
\date{\today}
\title{}

\hypersetup{
 pdfauthor={Lokesh Mohanty},
 pdftitle={},
 pdfkeywords={},
 pdfsubject={},
 pdfcreator={Emacs 29.0.60 (Org mode 9.7-pre)}, 
 pdflang={English}}
\begin{document}

\begin{frame}{Outline}
\tableofcontents
\end{frame}


\begin{frame}[label={sec:org8a6d250}]{Federated Learning}
\begin{itemize}
\item Data transfer, data privacy
\end{itemize}
\begin{block}{A more complex slide}
This slide illustrates the use of Beamer blocks.  The following text,
with its own headline, is displayed in a block:
\begin{theorem}[Org mode increases productivity]\label{sec:org8ac8df5}
\begin{itemize}
\item org mode means not having to remember \LaTeX{} commands.
\item it is based on ascii text which is inherently portable.
\item Emacs!
\end{itemize}

\hfill \(\qed\)
\end{theorem}
\end{block}
\end{frame}

\begin{frame}[label={sec:orgb07f38b}]{Hedonic Game (Game theoretic approach)}
\begin{itemize}
\item different distributions
\item join only for benefit
\end{itemize}
\end{frame}
\begin{frame}[label={sec:org117f25a}]{Model of federated learning}
\begin{itemize}
\item Error model for
\begin{itemize}
\item Uniform federation
\item Coarse-grained federation
\item Fine-grained federation
\end{itemize}
\end{itemize}

\begin{NOTES}
\begin{itemize}
\item Explain these different types of federation
\end{itemize}
\end{NOTES}
\end{frame}

\begin{frame}[label={sec:org56ea401}]{Stability}
\begin{block}{Uniform Federation}
\begin{itemize}
\item Same number of samples
\item Small/large number of samples
\end{itemize}
\end{block}

\begin{block}{Coarse-grained federation}
\begin{itemize}
\item Same number of samples
\item Small/large number of samples
\end{itemize}
\end{block}
\end{frame}

\begin{frame}[label={sec:org855faeb}]{Fine-grained federation}
\begin{columns}
\begin{column}{0.5\columnwidth}
\begin{itemize}
\item this is fine grained company
\item this is fine grained company
\item this is fine grained company
\end{itemize}
\end{column}
\begin{column}{0.5\columnwidth}
\begin{example}[Fine grained image]\label{sec:orge0ef8e5}
\begin{center}
\includegraphics[width=.9\linewidth]{/home/lokesh/Pictures/Screenshots/Screenshot from 2023-04-09 22-35-59.png}
\end{center}
\end{example}
\end{column}
\end{columns}
\end{frame}
\begin{frame}[label={sec:org78ed917}]{Optimality}
\begin{itemize}
\item Optimal: minimizes weighted sum of errors across all agents
\item Algorithm:
\begin{itemize}
\item Start with every agent doing local learning
\item Group the agents togetner in ascending order of size, stopping when the first agent would increase its error by joining the coalition
\end{itemize}
\item Equivalence of player preference and reducing cost
\end{itemize}

\begin{NOTES}
\begin{itemize}
\item Error model taken from the paper "Model sharing games" by the same author
\end{itemize}
\end{NOTES}

\begin{itemize}
\item Swapping
\item Monotonicity of joining
\item Monotonicity of leaving
\item Merging
\end{itemize}

\begin{itemize}
\item Model
\[ \hat{\theta}_C = \frac{1}{\sum_{i \in C} n_i} . \sum_{i \in C} n_i . \hat{\theta}_i\]
\[ err_j(C) = \frac{\mu_e}{\sum_{i \in C}n_i} + \sigma^2 . \frac{\sum_{i \in C, i \neq j} n_i^2 + \left( \sum_{i \in C, i \neq j} n_i \right)^2}{\left( \sum_{i \in C} n_i \right)^2}\]
\end{itemize}
\end{frame}
\begin{frame}[label={sec:org2ba618f}]{Price of Anarchy}
\begin{itemize}
\item for \(n_i \geq \frac{\mu_e}{\sigma^2}, \forall i\), the grand coalition \(\pi_g\) is always core stable
\item for \(n_i \leq \frac{\mu_e}{\sigma^2}, \forall i\), the individually stable or core stable is also optimal
\item \(\Pi_M\) is maximum cost IS partition, then \(err_i(\Pi_M) \leq \frac{\mu_e}{n_i}\) for all players \(i\)
\end{itemize}

\begin{itemize}
\item Error lower bound when a player \(j\) joins coalition \(C\)
\[err_j(C \cup \{n_j\}) \geq \begin{cases} \frac{1}{2}.\frac{\mu_e}{n_j}, n_j \geq \frac{\mu_e + \sigma^2}{2\sigma^2} \\ \sigma^2, \text{otherwise}\end{cases}\]
\item Error upper bound when a player \(j\) joins coalition \(C\) if total number of samples \(N_C \geq \frac{\mu_e}{3\sigma^2}\), then
\[err_j(C \cup \{n_j\} \leq 7.25.\sigma^2\]
\end{itemize}

\begin{itemize}
\item \(n_i \leq \frac{\mu_e}{3\sigma^2}, \forall i\), with atleast one player in a coalition with mass of its partners no more than \(\frac{\mu_e}{3\sigma^2}\), then the only stable arrangement of these players is to have all of them federating together
\item Price of Anarchy
\[ PoA = \frac{f_w(\Pi_M)}{f_w(\Pi_{opt})} \leq 9\]
\end{itemize}
\end{frame}
\begin{frame}[label={sec:org4adb6a4}]{Limitations}
\begin{itemize}
\item Theoretical study, results might be different in practice
\item Optimality bound has some assumptions, this may lead to different bounds
\end{itemize}
\end{frame}
\begin{frame}[label={sec:orgabc9924},fragile]{Related Work}
 \begin{itemize}
\item \texttt{Donahue and Kleinberg} studied models of fairness
\item \texttt{Hu et al. 2023} models clients behaviour in network
\item \texttt{Cui et al 2021} tries to find collaboration equilibrium
\item \texttt{Le et al. 2021} analyzes incentives for agents to contribute computational resources while using an auction approach
\end{itemize}
\end{frame}
\begin{frame}[label={sec:org3f00a2e}]{My extension}
\begin{itemize}
\item Nothing as of now
\end{itemize}
\end{frame}
\begin{frame}[label={sec:orga37b084}]{Conclusion}
Thank You
\end{frame}
\end{document}